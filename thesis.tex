\documentclass[runningheads,a4paper]{llncs}

%cmap has to be loaded before any font package (such as cfr-lm)
\usepackage{cmap}
\usepackage[T1]{fontenc}

\usepackage{graphicx}
\graphicspath{{./figures/}}

%Even though `american`, `english` and `USenglish` are synonyms for babel package (according to https://tex.stackexchange.com/questions/12775/babel-english-american-usenglish), the llncs document class is prepared to avoid the overriding of certain names (such as "Abstract." -> "Abstract" or "Fig." -> "Figure") when using `english`, but not when using the other 2.
%english has to go last to set it as default language
\usepackage[ngerman,english]{babel}
%Hint by http://tex.stackexchange.com/a/321066/9075 -> enable "= as dashes
\addto\extrasenglish{\languageshorthands{ngerman}\useshorthands{"}}

%better font, similar to the default springer font
%cfr-lm is preferred over lmodern. Reasoning at http://tex.stackexchange.com/a/247543/9075
\usepackage[%
rm={oldstyle=false,proportional=true},%
sf={oldstyle=false,proportional=true},%
tt={oldstyle=false,proportional=true,variable=true},%
qt=false%
]{cfr-lm}
%
%if more space is needed, exchange cfr-lm by mathptmx
%\usepackage{mathptmx}

%for demonstration purposes only
\usepackage[math]{blindtext}

%Sorts the citations in the brackets
%It also allows \cite{refa, refb}. Otherwise, the document does not compile.
%  Error message: "White space in argument"
\usepackage{cite}


%% If you need packages for other papers,
%% START COPYING HERE
%% COPY ALSO cmap and fontenc from lines 10 to 12

%extended enumerate, such as \begin{compactenum}
\usepackage{paralist}

%put figures inside a text
%\usepackage{picins}
%use
%\piccaptioninside
%\piccaption{...}
%\parpic[r]{\includegraphics ...}
%Text...

%for easy quotations: \enquote{text}
\usepackage{csquotes}

%enable margin kerning
\usepackage{microtype}

%tweak \url{...}
\usepackage{url}
%\urlstyle{same}
%improve wrapping of URLs - hint by http://tex.stackexchange.com/a/10419/9075
\makeatletter
\g@addto@macro{\UrlBreaks}{\UrlOrds}
\makeatother
%nicer // - solution by http://tex.stackexchange.com/a/98470/9075
%DO NOT ACTIVATE -> prevents line breaks
%\makeatletter
%\def\Url@twoslashes{\mathchar`\/\@ifnextchar/{\kern-.2em}{}}
%\g@addto@macro\UrlSpecials{\do\/{\Url@twoslashes}}
%\makeatother

%diagonal lines in a table - http://tex.stackexchange.com/questions/17745/diagonal-lines-in-table-cell
%slashbox is not available in texlive (due to licensing) and also gives bad results. This, we use diagbox
%\usepackage{diagbox}

%required for pdfcomment later
\usepackage{xcolor}


%enable nice comments
%this also loads hyperref
\usepackage{pdfcomment}
%enable hyperref without colors and without bookmarks
\hypersetup{hidelinks,
   colorlinks=true,
   allcolors=black,
   pdfstartview=Fit,
   breaklinks=true}
%enables correct jumping to figures when referencing
\usepackage[all]{hypcap}

\newcommand{\commentontext}[2]{\colorbox{yellow!60}{#1}\pdfcomment[color={0.234 0.867 0.211},hoffset=-6pt,voffset=10pt,opacity=0.5]{#2}}
\newcommand{\commentatside}[1]{\pdfcomment[color={0.045 0.278 0.643},icon=Note]{#1}}

%compatibality with packages todo, easy-todo, todonotes
\newcommand{\todo}[1]{\commentatside{#1}}
%compatiblity with package fixmetodonotes
\newcommand{\TODO}[1]{\commentatside{#1}}

%enable \cref{...} and \Cref{...} instead of \ref: Type of reference included in the link
\usepackage[capitalise,nameinlink]{cleveref}
%Nice formats for \cref
\crefname{section}{Sect.}{Sect.}
\Crefname{section}{Section}{Sections}

\usepackage{xspace}
%\newcommand{\eg}{e.\,g.\xspace}
%\newcommand{\ie}{i.\,e.\xspace}
\newcommand{\eg}{e.\,g.,\ }
\newcommand{\ie}{i.\,e.,\ }

%introduce \powerset - hint by http://matheplanet.com/matheplanet/nuke/html/viewtopic.php?topic=136492&post_id=997377
\DeclareFontFamily{U}{MnSymbolC}{}
\DeclareSymbolFont{MnSyC}{U}{MnSymbolC}{m}{n}
\DeclareFontShape{U}{MnSymbolC}{m}{n}{
    <-6>  MnSymbolC5
   <6-7>  MnSymbolC6
   <7-8>  MnSymbolC7
   <8-9>  MnSymbolC8
   <9-10> MnSymbolC9
  <10-12> MnSymbolC10
  <12->   MnSymbolC12%
}{}
\DeclareMathSymbol{\powerset}{\mathord}{MnSyC}{180}

% correct bad hyphenation here
\hyphenation{op-tical net-works semi-conduc-tor}

%% END COPYING HERE


\begin{document}

\title{Testing in Production for a fewer Risk.}
%If Title is too long, use \titlerunning
%\titlerunning{Short Title}

%Single insitute
\author{Sebastian Schasse}
%If there are too many authors, use \authorrunning
%\authorrunning{First Author et al.}
\institute{
  \email{schasse@mail.tu-berlin.de}\\
  Technische Universität Berlin
}

%Multiple insitutes
%Currently disabled
%
\iffalse
%Multiple institutes are typeset as follows:
\author{Firstname Lastname\inst{1} \and Firstname Lastname\inst{2} }
%If there are too many authors, use \authorrunning
%\authorrunning{First Author et al.}

\institute{
Insitute 1\\
\email{...}\and
Insitute 2\\
\email{...}
}
\fi

\maketitle

\begin{abstract}
Abstract goes here
\end{abstract}

\begin{keywords}
keyword1, keyword2
\end{keywords}

%%%%%%%%%%%%%%%%%%%%%%%%%%%%%%%%%%%%%%%%%%%%%%%%%%%%%%%%%%%%%%%%%%%%%%%%%%%%%%%
\section{From Blue-Green Deployments to Phoenix Replacement}

The approach of blue green deployment~\cite[find page]{continuous_delivery} involves two environments. One is the environment which serves production traffic, the other environment is in standby. You deploy to the other environment the new version. You then switch routing from the environment, which runs the old version, to the environment with the new version.

This has the major disadvantage that it is a waste of resources. Just one environment doing work with serving production traffic as the other environment is just idling. Even if you use the idling environment as a staging it would be oversized and still wasting
resources.

In times of dynamic resource allocation, automation and virtual machines, it became easy to automatically spawn new servers. In for a deploy of a new version you would not change the servers, but automatically create new ones with the version to deploy, then switch the traffic from the old servers to the new servers, and in the end just destroy the old servers. This procedure is called phoenix replacement~\cite[p. 283]{infra_as_code}.

\section{Canary Releases}

Canary releasing~\cite[find page]{continuous_delivery} is a way to test new versions of the application in production. But testing in production is risky, because when there is an error which leads to a defect or failure the users will get affected. And this costs money in any way.

There comes canary releasing into play. A single canary can't do much harm and it's not a big catastrophe if the small canary is malicious. Let me explain it at the example of a typically scaled web application. Usually you do not have just a single web server, but multiple servers. So when releasing in the canary style, you change just a small proportion of the twenty webservers to the new version. Now two versions of the webservers are running at the same time.

Usually you want exactly the same version of webservers in production with the exact same configuration. This makes systems easier to debug in case of an error. The maximum count of different versions during canary releasing is two. But why go for two different versions in production with the canary releasing technique, when it makes debugging in general harder.

It's because with canary releasing you can achieve different goals. The first one is truly when an error occurs. Just because less users are affected by the error. The majority of webservers is still in the old version, so just a small portion of all the users have a poor experience with the erroneous small canary version fraction.

In case of success, when everything works as expected, it is proven that there is less risk of errors, even under real production conditions.

canary releasing with features

canary releasing when doing refactorings.

\section{Deployments}

The extend to the canary releasing technique is a rolling update. It unifies phoenix replacement and canary releasing and extends it to a full deployment. You start with one server in the new version and spawn it automatically. Now it becomes part of the whole pool of servers and serves traffic as the server is ready. Now two different versions are serving traffic, just like with a canary release. But after the first step the deployment goes on and then destroys a server instance in the old version as you would do in the phoenix replacement. Then the procedure will be done multiple times until all servers of the old version are replaced by the new version.

%%%%%%%%%%%%%%%%%%%%%%%%%%%%%%%%%%%%%%%%%%%%%%%%%%%%%%%%%%%%%%%%%%%%%%%%%%%%%%%
\bibliographystyle{splncs03}
\bibliography{thesis}

All links were last followed on October 5, 2014.
%%%%%%%%%%%%%%%%%%%%%%%%%%%%%%%%%%%%%%%%%%%%%%%%%%%%%%%%%%%%%%%%%%%%%%%%%%%%%%%

\end{document}
