\newcommand{\myterm}[1]{#1}
\newcommand{\myemph}[1]{#1}

\newcommand{\nprt}{\myterm{Nonfunctional Production Regression Testing}}
\newcommand{\deployer}{\myterm{Deployer}}
\newcommand{\gemupdater}{\myterm{GemUpdater}}
\newcommand{\depctl}{\myterm{depctl}}

\newglossaryentry{g:vcs}{
  name=version control,
  description={is a practice where developers keep track of every change of their
    code. A repository stores every version.
    Examples for version control systems are Subversion and Git.
    Humble explains what is important in version control~\cite{cd_humble_config}}
}
\newglossaryentry{g:ci}{
  name=continuous integration,
  description={is a practice of continuously merging code into a
    repository and automatically testing it~\cite{ci_fowler}.
    A continuous integration system is a server that automates tests}
}
\newglossaryentry{g:devops}{
  name=DevOps,
  description={is made up of developers and operations.
    DevOps practices aim for a short time between committing code and deployment while
    still ensuring high quality~\cite{devops_definition}}
}
\newglossaryentry{g:microservices}{
  name=microservices,
  description={are a software architecture in which several small services
  create a modularized software system, each with a specific function. These
  services communicate with each other via a network~\cite{microservices_fowler}}
}
\newglossaryentry{g:cd}{
  name=continuous delivery,
  description={is a practice to frequently deliver changes to production.
    Continuous delivery is related to DevOps~\cite{cd_wolff_devops}}
}
\newglossaryentry{g:cdep}{
  name=continuous deployment,
  description={is an extension of continuous delivery. An automated deploy process
    delivers the changes directly to production~\cite{cd_humble_deploy}}
}
\newglossaryentry{g:canary}{
  name=canary releasing,
  description={is a releasing practice to reduce risk. A canary serves as the test instance
    for a change in production~\cite{cd_humble_deploy}}
}
\newglossaryentry{g:zdp}{
  name=zero downtime deploys,
  description={is a way of deploying changes without
    downtime~\cite{cd_humble_deploy}. Rolling updates is an implementation}
}
\newglossaryentry{g:mon}{
  name=monitoring system,
  description={collects information about a software product and monitors it for
    defects. I refer in the thesis to a monitoring system with a modern design as Ewaschuk and Bass
    et.~al describe~\cite{sre_monitoring,devops_monitoring}}
}
\newglossaryentry{g:dynamic}{
  name=dynamic infrastructure,
  description={is software-defined
  infrastructure~\cite{infra_as_code_platforms,infra_as_code_se_practices}}
}
\newglossaryentry{g:immutable}{
  name=immutable infrastructure,
  description={in order to change instances in such an infrastructure,
  nodes are destroyed and created}
}
\newglossaryentry{g:kubernetes}{
  name=Kubernetes,
  description={is a cluster management system which was heavily influenced by Google's Borg
    ~\cite{borg,borg_kubernetes}. Beda gives a good
    overview of the architecture~\cite{kubernetes_architecture2}}
}
\newglossaryentry{g:container}{
  name=container,
  description={is a running instance of a container image~\cite{docker_orientation,docker_orientation2}}
}
\newglossaryentry{g:image}{
  name=container image,
  description={is an immutable snapshot of a container state~\cite{docker_orientation,docker_orientation2}}
}
\newglossaryentry{g:pod}{
  name=Pod,
  description={is a Kubernetes resource. A Pod consists of multiple containers
 located on the same host that are able to share resources~\cite{pod}}
}
\newglossaryentry{g:service}{
  name=Service,
  description={is a Kubernetes resource that acts as a load balancer
  in front of Pods~\cite{service}}
}
\newglossaryentry{g:deployment}{
  name=Deployment,
  description={is a Kubernetes resource that ensures the existence of replicated
    Pods. Deployments enable the implementation of stateless instances such as web servers and
    provide rolling updates. Deployments are an evolution to ReplicationControllers,
    which had similar features~\cite{replicationcontroller}}
}
\newglossaryentry{g:statefulset}{
  name=StatefulSet,
  description={is a Kubernetes resource similar to Deployments. However, the StatefulSet
  creates Pods with fixed hostnames and eventually binds them to fixed volumes. The
  StatefulSet enables the realization of distributed, stateful applications such as databases
  }
}
\newglossaryentry{g:resource}{
  name=resource definitions,
  description={are files that define Kubernetes resources, such as Service,
    Deployment, Pod etc.}
}
\newglossaryentry{g:versionref}{
  name=version reference,
  description={is either a commit hash (Git) or revision (Subversion).
    In \gls{nprt} the container image tag includes a version reference}
}
\newglossaryentry{g:rolling}{
  name=rolling updates,
  description={is a mechanism to update multiple instances of a system without downtime.
  One instance will be created in the new version, and an instance of the old version
  will be shut down. The mechanism proceeds until every instance is in the new version}
}
\newglossaryentry{g:registry}{
  name=container image registry,
  description={is a repository that stores container images.}
}


\newacronym{nprt}{\myterm{NPRT}}{\nprt{}}
\newacronym{cd}{CD}{Continuous Delivery}
\newacronym{ci}{CI}{Continuous Integration}
\newacronym{vcs}{VCS}{Version Control System}
\newacronym{din}{DIN}{Deutsches Institut für Normung}
\newacronym{sre}{SRE}{Site Reliability Engineering}
\newacronym{cpu}{CPU}{Central Processing Unit}
\newacronym{api}{API}{Application Programming Interface}
\newacronym{http}{HTTP}{Hyper Text Transfer Protocol}
