\newcommand{\myterm}[1]{\textit{#1}~}
\renewcommand{\emph}[1]{\textbf{#1}}

\newcommand{\nprt}{\myterm{Nonfunctional Production Regression Testing}}
\newcommand{\deployer}{\myterm{Deployer}}
\newcommand{\gemupdater}{\myterm{GemUpdater}}
\newcommand{\depctl}{\myterm{depctl}}

\newglossaryentry{g:vcs}{
  name=version control,
  description={is a practice, where developers keep track of every change of their
    code. A repository stores every version.
    Examples for version control systems are subversion and git.
    Humble explains what is important in version control~\cite{cd_humble_config}}
}
\newglossaryentry{g:ci}{
  name=continuous integration,
  description={is a practice of continuously merging code to a
    repository and automatically test it~\cite{ci_fowler}.
    A continuous integration system is a server, which automates the tests.}
}
\newglossaryentry{g:devops}{
  name=DevOps,
  description={is a made up of developers and operations.
    Devops practices aim for a short time between commit and deploy,
    still ensuring high quality~\cite{devops_definition}}
}
\newglossaryentry{g:microservices}{
  name=microservices,
  description={TODO}
}
\newglossaryentry{g:cd}{
  name=continuous delivery,
  description={is a practice to continuously deliver changes to production.
    Continuous delivery is related to DevOps~\cite{cd_wolff_devops}}
}
\newglossaryentry{g:cdep}{
  name=continuous deployment,
  description={is an extension to continuous delivery. An automated deployment process,
    which delivers most changes directly to production~\cite{cd_humble_deploy}}
}
\newglossaryentry{g:canary}{
  name=canary releasing,
  description={is a releasing practice to reduce risk. A canary serves as test instance
    for a change in production~\cite{cd_humble_deploy}}
}
\newglossaryentry{g:zdp}{
  name=zero downtime deploys,
  description={is a way of deploying changes without
    downtime~\cite{cd_humble_deploy}. Rolling updates is an implementation}
}
\newglossaryentry{g:mon}{
  name=monitoring system,
  description={collects information about a software product and monitors it for
    defects. I refer in the thesis to modern monitoring system design as Ewaschuk and Bass
    et.~al describe~\cite{sre_monitoring,devops_monitoring}}
}
\newglossaryentry{g:dynamic}{
  name=dynamic infrastructure,
  description={is software defined infrastructure~\cite{infra_as_code_platforms,infra_as_code_se_practices}}
}
\newglossaryentry{g:immutable}{
  name=immutable infrastructure,
  description={In such an infrastructure nodes are destroyed and created in order to
    change them}
}
\newglossaryentry{g:kubernetes}{
  name=Kubernetes,
  description={is a cluster management system, which Google's borg
    heavily influenced~\cite{borg,borg_kubernetes}. Beda gives a good
    overview of the architecture~\cite{kubernetes_architecture2}}
}
\newglossaryentry{g:container}{
  name=container,
  description={is a running instance of a container image~\cite{docker_orientation,docker_orientation2}}
}
\newglossaryentry{g:image}{
  name=container image,
  description={is an immutable snapshot of a container state~\cite{docker_orientation,docker_orientation2}}
}
\newglossaryentry{g:pod}{
  name=pod,
  description={is a kubernetes resource. It is a unit of multiple containers,
    which are located on the same host to be able to share resources~\cite{pod}}
}
\newglossaryentry{g:service}{
  name=service,
  description={is a kubernetes resource, which basically acts as a loadbalancer in front
  of pods~\cite{service}}
}
\newglossaryentry{g:deployment}{
  name=deployment,
  description={is a kubernetes resource, which ensures the existence of replicated
    pods. Deployments are an evolution to replicationcontrollers, which have similar
    features~\cite{replicationcontroller}}
}
\newglossaryentry{g:statefulset}{
  name=statefulset,
  description={TODO}
}
\newglossaryentry{g:deploy}{
  name=deploy,
  description={I use the term deploy in the thesis in order to differentiate the procedure
    from deployment, the kubernetes resource.}
}
\newglossaryentry{g:resource}{
  name=resource definitions,
  description={are files, which define kubernetes resources such as service,
    deployment, pod etc..}
}
\newglossaryentry{g:versionref}{
  name=version reference,
  description={is either a commit hash (git), revision (subversion).
    In \gls{nprt} the container image tag includes a version reference.}
}
\newglossaryentry{g:rolling}{
  name=rolling updates,
  description={TODO}
}
\newglossaryentry{g:registry}{
  name=container image registry,
  description={TODO}
}

\newacronym{nprt}{\myterm{NPRT}}{\nprt}
\newacronym{cd}{CD}{Continuous Delivery}
\newacronym{ci}{CI}{Continuous Integration}
\newacronym{vcs}{VCS}{Version Control System}
\newacronym{din}{DIN}{Deutsches Institut für Normung}
\newacronym{sre}{SRE}{Site Reliability Engineering}

% slack, ruby on rails etc... linux
