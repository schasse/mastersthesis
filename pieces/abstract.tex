\section*{Abstract}
Many companies successfully use DevOps practices and microservices to eliminate the
conflicts between developers and the operation team. Initially, the thesis explains one of
the impacts of those practices: a shift of responsibility of maintenance work from
operations to development. Developers have to maintain their own code as well as many
dependencies. Containerization technologies consolidate this shift. A problem is that the
continuous delivery pipeline supports developers only up until the deploy and no
further. \gls{nprt} extends the continuous delivery pipeline to the production
environment. It addresses the issue of maintenance in a continuous and fully automatable
fashion. In \gls{nprt}, a new version in form of a canary will be automatically deployed
to production and compared to the stable version. In case of a regression the canary will
be automatically rolled back. In order to enable \gls{nprt} with Kubernetes, I implemented
\deployer which connects the continuous integration system with a monitoring system and
Kubernetes. An Apache middleware, which is used at \gls{din}, and a Rails application,
which GapFish develops and runs in production, serve in multiple experiments and in order
to evaluate \gls{nprt} for real life use cases.
\clearpage

\section*{Zusammenfassung}
Viele Firmen setzen DevOps Praktiken und Microservices ein, um die Konflikte zwischen den
Entwlickern und dem Operation Team zu eliminieren. Zunächst erklärt die Masterarbeit eine
der Auswirkungen dieser Praktiken: eine Verlagerung der Verantwortung der Wartungsarbeiten
von Betrieb nach Entwicklung. Entwickler müssen sowohl ihren eigenen Code als auch viele
Abhängigkeiten warten. Containerization Technologien verfestigen diese Verlagerung. Ein
Problem ist, dass die Continuous Delivery Pipeline Entwickler lediglich bis zum Deploy
unterstützt und nicht weiter. \gls{nprt} erweitert die Continuous Delivery Pipeline bis in
die Produktionsumgebung. Es ist ein kontinuierlich anwendbarer und voll automatisierter
Ansatz, der das Problem der Wartungsarbeiten addressiert. Bei \gls{nprt} wird eine neue
Version in Form eines Canarys automatisch in Produktion deployt und mit der stabilen
Version verglichen. Im Fall einer Regression wird der Canary automatisch zurückgerollt. Um
\gls{nprt} mit Kubernetes zu ermöglichen, habe ich \deployer implementiert, der das
Continuous Integration System mit dem Monitoring System und Kubernetes verbindet. Ein
Apache Middlewareserver, der bei \gls{din} verwendet wird, und eine Rails Anwendung, die
von GapFish entwickelt und in Produktion betrieben wird, dienen in mehreren Experimenten,
um \gls{nprt} praxisnah zu evaluieren.
