\part{Foundations}
\chapter{Introduction: Developer and operation teams converge and both use software engineering practices}

\chapter{Developers use the Continuous Delivery Pipeline}
\section{The Continuous Delivery Pipeline consists of commitment, continuous integration and deployment}
\section{Software Deployment approaches evolved from manual to automated}
\subsection{Blue-Green Deployment allows Zero Downtime releases}
\subsection{Automation leads to resource saving Phoenix Deployment and Rolling Deployments}
\subsection{Canaries test releases with a small amount of traffic}
\subsection{continuous deployment is not continuous delivery}

\chapter{Operators evolved to Site Reliability Engineers}
\section{Site Reliability Engineers maintain applications like software engineers}
sw installation, hw installation, logging, scaling, monitoring (detecting problems), security, incident management, support
\section{Monitoring to identify Problems}
\subsection{Health checks measure availability}
\subsection{Measuring Latency, Traffic, Errors and Saturation identifies failures and performance problems}
\subsection{Incident Management (/Notifications) for appropriate and fast actions in case of Problems}

\part{New Practices}
\chapter{Post Release Testing extends the Continuous Delivery Pipeline to support maintaining a system}
\section{Post Release Testing leads to lower time to market}
\section{It makes Releases consistent, measurable, fast and scalable}
mttr
\section{It is a new opportunity for risk management}
identify test before release is a mttr of zero, after release still fast. easier to test in production (complexity of system)

\section{Companies are already post release testing their software systems}
\subsection{Netflix uses Simian Army to live test their systems}
\subsection{Synthetic Monitoring tests a complex distributed system}

\section{Non Change Post Release Testing with Canaries}
\subsection{black box monitoring is only one part and monitoring change is difficult}
\subsection{Continuous Delivery is a requirement}
\subsection{Notifications in case a canary behaves different}
\subsection{Automated Rollbacks for a automatic self healing system}

\chapter{Implementing Canary Post Release Testing}
\section{New technologies drive new techniques}
\subsection{Kubernetes is a Cluster OS}
\subsubsection{overview: cluster os - resource management}
\subsubsection{immutability}
\subsection{datadog}
\subsection{ruby - sinatra}

\section{the deployer}
\subsection{architecture and how it integrates in the pipeline}
\subsection{importance for maintenance and feature deploys}
\subsection{cycle time vs. notifications}

\chapter{Velocity and cycletime as efficiency metrics of an agile team}
\chapter{MTTR and ErrorRate measures the quality of a software}

% \subsection{bugs + debugging}
% \subsection{ilities}
% \subsection{security}
% \subsection{monitoring}

\part{Evaluation}
\chapter{GapFish a startup company}
\chapter{GapFish's traditional toolchain and teams}
\chapter{team agility metrics}
\section{cycle time measures quality of delivery engine}
\section{locs/deploy measures risk}
\section{deploys per day measures agility}
\chapter{software quality metrics}
\section{ErrorRate as monitoring measure for automation}
\section{problems in error rate measure defect and failure}
\section{solution a secific heuristic}
\chapter{Results}
\section{traditional vs. new}
\chapter{Lessons learned and future}

\chapter{theoretical/practical conclusion}
\chapter{for Gapfish}
