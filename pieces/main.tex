\part{Foundations}

\chapter{Introduction: Developer and operation teams converge and both use software engineering practices}

\chapter{Developers use the Continuous Delivery Pipeline}
\section{The Continuous Delivery Pipeline consists of commitment, continuous integration and deployment}
\section{Software Deployment approaches evolved from manual to automated}
\subsection{Blue-Green Deployment allows Zero Downtime releases}
\subsection{Automation leads to resource saving Phoenix Deployment and Rolling Deployments}
\subsection{Canaries test releases with a small amount of traffic}
\subsection{continuous deployment is not continuous delivery}

\chapter{Operators evolved to Site Reliability Engineers}
\section{Site Reliability Engineers maintain applications like software engineers}
sw installation, hw installation, logging, scaling, monitoring (detecting problems), security, incident management, support
\section{Monitoring to identify Problems}
\subsection{Health checks measure availability}
\subsection{Measuring Latency, Traffic, Errors and Saturation identifies failures and performance problems}
\subsection{Incident Management (/Notifications) for appropriate and fast actions in case of Problems}

\chapter{Metrics can indirectly measure team efficiency and software quality}
\section{Velocity and cycletime are efficiency metrics for an agile team}
cycle time measures quality of delivery engine
\subsection{Deploys/Week indirectly measures velocity}
and it measures the effect of a quality delivery engine
we want many deploys per week
\subsection{Deploy Duration is import for cycle time}
\section{MTTR and Failurerate measures the quality of a software}
we want low risk per deploy to achieve MTTR and low Failure Rate
\subsection{LOCS/Deploy indirectly measures the risk per Deploy}


\part{New Practices}

\chapter{Post Release Testing extends the Continuous Delivery Pipeline to support maintaining a system}
\section{Post Release Testing leads to lower time to market}
cycle time
\section{It makes Releases consistent, measurable, fast and scalable}
mttr, automation/automatic discussion
\section{It is a new opportunity for risk management}
identify test before release is a mttr of zero, after release still fast. easier to test in production (complexity of system)
\section{Companies are already post release testing their software systems}
\subsection{Netflix uses Simian Army to live test their systems}
\subsection{Synthetic Monitoring tests a complex distributed system}
\section{Post Release Testing with Canaries is appropriate for testing non change}
ErrorRate as monitoring measure for automation
problems in error rate measure defect and failure
solution a secific heuristic
\subsection{Black-Box monitoring is only one part and monitoring change is difficult}
\subsection{Canary testing is important for maintenance but not feature deploys}
\subsection{Continuous Delivery is a requirement}
\subsection{Notifications in case a canary behaves different}
\subsection{Automated Rollbacks for a automatic self healing system}

\chapter{Implementing Canary Post Release Testing}
\section{New technologies drive new techniques}
\subsection{Kubernetes is a Cluster OS}
\subsubsection{Resource Management in Kubernetes is made for high available services}
\subsubsection{Deployments implement Rolling Updates}
\subsection{DataDog is a Cluster Monitoring Systems as SaaS}
alternativen Graphana, Prometheus
\subsubsection{The main components are: Datacollection, Timeseriesdatabase, Graphing and Alerting}
\subsubsection{DataDog integrates well in Kubernetes}
\section{Deployer is a Service for Continuous Deployment and enables Canary Post Release testing}
\subsection{Deployer integrates into the Continuous Delivery Pipeline}
\subsection{Deployer integrates into Kubernetes and deploys itself}
\subsection{Deployer integrates into Monitoring and enables Canary testing}
\subsection{The main Deployer API features are Deployments and Canary deployments}
depctl, curl
\subsubsection{Deployments deploy a whole repository}
\subsubsection{One Canary per Replicated Pods can be deployed and monitored}
\subsubsection{Immediate Notifications in case of failure}
difference depctl and curl
\subsubsection{Automatic Rollbacks in case defect Canary}
via datadog and triggers deployer
future: staging deploys

% \subsection{bugs + debugging}
% \subsection{ilities}
% \subsection{security}
% \subsection{monitoring}


\part{Evaluation}

\chapter{GapFish is the company to evaluate Deployer}
\section{GapFish's services are complex and highly available}
overview of Gapfish's services
\subsection{GapFish's Operation Service is used by internal Staff and Customers}
operation service == operation.gapfish.com
\section{GapFish uses tools for continuous deployment}
\subsection{GapFish differentiates between development and operation}
how is the deployment tradditionally done
\subsection{Kubernetes enables GapFish to have a development and operation in one team}
which services are migrated to kubernetes

\chapter{Implementation of Metrics: traditional vs new}
\section{Deploys/Week}
\section{Deploy Duration}
\section{LOCS/Deploy}

\chapter{Results}
\section{Metrics: traditional vs. new}
\section{theoretical/practical conclusion for deployer and cd}
\section{for Gapfish}
\section{Lessons learned and future}

\chapter{resume}
